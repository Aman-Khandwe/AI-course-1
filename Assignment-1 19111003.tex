\documentclass{article}
\usepackage[utf8]{inputenc}
\usepackage[margin=1.5in]{geometry}

\title{\textbf{Philosophy of Articial Intelligence}}
\author{Ajay Kumar Sahu }
\date{20 July 2021}

\begin{document}

\maketitle

\section{Introduction}

Philosophy of Artificial intelligence investigates the artificial intelligence and it's features to show \textbf{knowledge}, \textbf{intelligence},\textbf{ethics}, \textbf{consciousness}, \textbf{epistemology}, and \textbf{free will}.As an agent mimic human intelligence it becomes a subject to grab the attention of philosophers. The philosophy of intelligence try to answer how much humanly character a machine or intelligent agent can posses in terms of thinking, \textbf{problem solving}, consciousness and other aspects. Some important propositions are given in the philosophy of AI to describe an intelligent system which are: are-Turing's "polite convention", The Dartmouth proposal,John Searle's strong AI hypothesis,Allen Newell and Herbert A.Simon's physical symbol system hypothesis,Hobbes' mechanism.

\section{Can a machine display general intelligence?}
The scope of intelligent systems in the future will be determined by the answers to this query. The proposal of Dartmouth workshop give the basic justification to the question which is, "\textit{Every aspect of learning or any other feature of intelligence can be so precisely described that a machine can be made to simulate it.}" Next step to answer this question is to answer, "What is \textbf{intelligence}?" which is described by Turing as,\textit{if a machine can answer any question put to it, using the same words that an ordinary person would, then we may call that machine intelligent. }Present day researcher define intelligence in terms of intelligent agent as an "\textbf{agent}" is something which perceives and acts in an environment. And "\textit{If an agent acts so as to maximize the expected value of a performance measure based on past experience and knowledge then it is intelligent.}" To support the argument Hubert Dreyfus described that the brain can be simulated as the nervous system follow laws which can be replicated. Adding to this Newell and Simon proposed that \textbf{symbol processing} is the essence of intelligence. \textit{"A physical symbol system has the necessary and sufficient means of general intelligent action."} To which Kurt Godel argued with incompleteness theorem that there's always a \textit{Godel statement} which an intelligent system could not prove. Another argument against the general intelligence was given by Dreyfus that \textit{human intelligence depends on implicit skills which can not be described by formal rules}.

\section{Can a machine have a mind, consciousness, and mental states?}
To address this, John Searle proposed the terminology "\textbf{strong AI}" and "\textbf{weak AI}." A strong AI is a physical symbol system which can have a \textbf{mind} and mental states whereas a weak AI is a physical symbol system which can act intelligently. But this definitions of Searle couldn't answer this optly. To which Russell and Norvig said: "Most AI researchers take the weak AI hypothesis for granted, and don't care about the strong AI hypothesis."
Different thinkers, philosopher and scientist defined mind and \textbf{consciousness} as something which make human what they are and doubt that it can be replicated in machine. Many arguments are given that computer can't have mind and mental state which are Searle's Chinese room, Leibniz' mill, Davis's telephone exchange, Block's Chinese nation and Blockhead.

\section{Is thinking a kind of computation?}
According to the computational theory of mind, the relationship between mind and brain is proportional to the running program and computer. And argument to this clears the practical and philosophical question of AI. As Hobbes wrote: "\textbf{Reasoning} is nothing but reckoning", which suggest that intelligence is a form of calculation which justify AI is possible. And by the claim of Stevan Harnad that "Mental states are just implementations of (the right) computer programs" suggest that AI can have consciousness.
\\

Thus philosophy of AI argue on different aspect of AI, which includes; general intelligence, consciousness, emotion, thinking, awareness, human behaviour etc. And acc to many scholors the role of philosophy and it's understanding is very important in development of AI.

\end{document}
